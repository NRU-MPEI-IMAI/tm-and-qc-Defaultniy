\documentclass{article}
\usepackage{cmap}
\usepackage[utf8x]{inputenc}
\usepackage[russian]{babel}
\usepackage{mathtext}
\usepackage[pdftex]{graphicx}
\usepackage[T2A]{fontenc}
\title{"Теоретические модели вычислений"}
\author{Артем Поляков А-13а-19}

\begin{document}

\maketitle
\newpage
\section{Задание №1. Машины Тьюринга}
\subsection{Операции с языками и символами}
\subsubsection{Сложение двух унарных чисел}

Алгоритм:

Движемся вправо пока не встретим знак сложения, найдя "+" заменяем его на "1" и движемся в конец, в конце стираем "1" и движемся к началу.

\includegraphics[width=18cm, height=8cm]{2_1_1.png}

\newpage
\subsubsection{Умножение двух унарных чисел}

Алгоритм:

Движемся вправо находим знак умножения, помечаем первую найденную единицу второго числа символом "х". Движемся влево, находя единицу первого множителя заменяем ее на "х" и движемся вправо и записываем ее в конец, пройдя все единицы первого множителя начинаем движение вправо заменяя символы "х" на "1", найдя следующую единицу второго множителя повторяем процесс, когда закончились единицы второго множителя, производим замену символов "х" второго множителя на "1".

\includegraphics[width=15cm, height=8cm]{2_1_2.png}

\newpage
\subsection{Операции с языками и символами}

\subsubsection{Принадлежность к языку \(L = \{0^n1^n2^n, n \geq 0 \}\)}

Алгоритм:

Заменяем первые вхождения "0", "1", "2" на "х". Возвращаемся в начало.

Повторяем предыдущий шаг, пока слово не будет заменено на все "х"(иначе слово не принадлежит языку)
1 - слово принадлежит языку, 0 - слово не принадлежит. n так же может быть равно нулю, такое слово тоже принадлежит языку.

\includegraphics[width=15cm, height=8cm]{2_2_1.png}
\newpage
\subsubsection{Проверка соблюдения правильности скобок в строке (минимум 3 вида) }

Алгоритм:
Ищем первую скобку. Меняем ее на "х". Возвращаемся в начало. Ищем открывающую скобку такого же вида, меняем ее на "x". Возвращаемся в начало. 1 - слово принадлежит языку (все "x"), 0 - нет. Как и в предыдущем номере, пустое слово - правильная скобочная последовательность.

\includegraphics[width=15cm, height=8cm]{2_2_2.png}

\end{document}
